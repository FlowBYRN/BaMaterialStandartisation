\begin{table}[h]

	\centering
\begin{tabular}{|l|l|l|l|}
	\hline
	\textbf{Begriff} & \textbf{Erwartet} & \textbf{DBSCAN} & \textbf{OpenAI}\\ \hline
   Ortbeton C30/37 Verputzt & 1 & 1 & ~ \\ \hline
Trockenbau Gipsplatte & ~ & ~ & ~ \\ \hline
Fußboden Heizestrich & 3 & 3 & 3 \\ \hline
Geländearbeiten Gebundene Schüttung & 4 & 4 & 4 \\ \hline
Ortbeton C30/37 & 1 & 1 & ~ \\ \hline
Mauerwerk mit Daemmeigenschaften & 2 & 2 & 2 \\ \hline
Fußboden Fußbodenaufbau & 3 & 3 & 3 \\ \hline
Fußboden Schüttung & 3 & ~ & 3 \\ \hline
Putz & ~ & ~ & ~ \\ \hline
Geländearbeiten Rollierung Schuettung & 4 & 4 & 4 \\ \hline
Ortbeton bewehrt geschliffen & 1 & ~ & ~ \\ \hline
Geländearbeiten Aufgeschüttet & 4 & 4 & 4 \\ \hline
Fußboden Estrich & 3 & 3 & 3 \\ \hline
Beton C 25/30 & 1 & 1 & ~ \\ \hline
Dachdeckung Ziegel & ~ & ~ & ~ \\ \hline
Mauerwerk ohne Daemmeigenschaften & 2 & 2 & 2 \\ \hline
\textbf{Auswertung} & richtig & teilweise & teilweise \\ \hline
	\end{tabular}
	\caption{Evaluationsbeispiel 1 (Überkategorie: Mineralisch)}
	\label{t:evaluation-example1}
\end{table}

\begin{table}[h]
	
	\centering
	\begin{tabular}{|l|l|l|l|}
		\hline
		\textbf{Begriff} & \textbf{Erwartet} & \textbf{DBSCAN} & \textbf{OpenAI}\\ \hline
		 Keramik & ~ & ~ & ~ \\ \hline
		 Trockenbau Gipsplatte & 1 & 1 & 1 \\ \hline
		 Trockenbau Gipsplatte 2 & 1 & 1 & 1 \\ \hline
		 Trockenbau Gipsplatte 3 & 1 & 1 & 1 \\ \hline
		 Trockenbau Gipsplatte 4 & 1 & 1 & 1 \\ \hline
		 Fußboden Teppich & 2 & ~ & 2 \\ \hline
		 Fußboden Estrich & 2 & ~ & 2 \\ \hline
		 Fußboden Trittschall & 2 & 2 & 2 \\ \hline
		 Fußboden Schüttung & 2 & 2 & 2 \\ \hline
		 Geschossdecke FB 200 Fliese & 3 & 3 & ~ \\ \hline
		 Fußboden Fliese Travertin & 2 & 3 & 2 \\ \hline
		 Ortbeton bewehrt & 3 & ~ & ~ \\ \hline
		 Mauerwerk Ziegel \\ \hline
		\textbf{Auswertung} & richtig & falsch & teilweise \\ \hline
	\end{tabular}
	\caption{Evaluationsbeispiel 2 (Überkategorie: Mineralisch)}
	\label{t:evaluation-example2}
\end{table}

\begin{table}[h]
	
	\centering
	\begin{tabular}{|l|l|l|l|}
		\hline
		\textbf{Begriff} & \textbf{Erwartet} & \textbf{DBSCAN} & \textbf{OpenAI}\\ \hline
      	Gipskarton & ~ & ~ & ~ \\ \hline
		Keramik Fliesen & ~ & ~ & ~ \\ \hline
		Mauerwerk Betonblock & 1 & 1 & 1 \\ \hline
		Beton Gegossen & 2 & 2 & 2 \\ \hline
		Mauerwerk Ziegel & 1 & 1 & 1 \\ \hline
		Beton & 2 & 2 & 2 \\ \hline	
		\textbf{Auswertung} & richtig & richtig & richtig \\ \hline
	\end{tabular}
	\caption{Evaluationsbeispiel 3 (Überkategorie: Mineralisch)}
	\label{t:evaluation-example3}
\end{table}

\begin{table}[h]
	
	\centering
	\begin{tabular}{|l|l|l|l|}
		\hline
		\textbf{Begriff} & \textbf{Erwartet} & \textbf{DBSCAN} & \textbf{OpenAI}\\ \hline
		 Stütze Auflager & ~ & ~ & -1 \\ \hline
		Kalksandstein & ~ & ~ & ~ \\ \hline
		Kies & ~ & ~ & ~ \\ \hline
		Fliesen & ~ & ~ & 2 \\ \hline
		Estrich & ~ & ~ & 2 \\ \hline
		Gipsputz & ~ & ~ & ~ \\ \hline
		Volkernplatte & ~ & ~ & 1 \\ \hline
		Ausbauplatte GKBI & 1 & 1 & 1 \\ \hline
		Ausbauplatte GKB & 1 & 1 & 1 \\ \hline
		\textbf{Auswertung} & richtig & richtig & falsch \\ \hline
	\end{tabular}
	\caption{Evaluationsbeispiel 4 (Überkategorie: Mineralisch)}
	\label{t:evaluation-example4}
\end{table}

\begin{table}[h]
	
	\centering
	\begin{tabular}{|l|l|l|l|}
		\hline
		\textbf{Begriff} & \textbf{Erwartet} & \textbf{DBSCAN} & \textbf{OpenAI}\\ \hline
		 WI KS 17 5 & 1 & 1 & ~ \\ \hline
		Kalksandstein & 1 & ~ & 1 \\ \hline
		WI KS 11 5 & 1 & 1 & 1 \\ \hline
		DA Flachdach Kiesschicht & 3 & ~ & ~ \\ \hline
		Kies & 3 & ~ & ~ \\ \hline
		Fliesen & ~ & ~ & 3 \\ \hline
		Estrich & ~ & ~ & 3 \\ \hline
		Gipsputz & ~ & ~ & ~ \\ \hline
		Ausbauplatte GKBI & 2 & 2 & 2 \\ \hline
		Ausbauplatte GKB & 2 & 2 & 2 \\ \hline
		\textbf{Auswertung} & richtig & teilweise & falsch \\ \hline
	\end{tabular}
	\caption{Evaluationsbeispiel 5 (Überkategorie: Mineralisch)}
	\label{t:evaluation-example5}
\end{table}

\begin{table}[h]
	
	\centering
	\begin{tabular}{|l|l|l|l|}
		\hline
		\textbf{Begriff} & \textbf{Erwartet} & \textbf{DBSCAN} & \textbf{OpenAI}\\ \hline
		  Mauerwerk Langlochziegel verputzt & 1 & 1 & 1 \\ \hline
		Mauerwerk Ziegel verputzt & 1 & 1 & 1 \\ \hline
		MW Kalksandstein verputzt & 1 & 1 & 1 \\ \hline
		Knauf Bauplatte Typ A & ~ & ~ & ~ \\ \hline
		Decke Fliesen & ~ & ~ & ~ \\ \hline
		Beton Estrich & ~ & ~ & ~ \\ \hline
		Porenbeton verputzt & 1 & 1 & ~ \\ \hline
		Vorgabe Dach & ~ & ~ & ~ \\ \hline
		\textbf{Auswertung} & richtig & richtig & teilweise \\ \hline
	\end{tabular}
	\caption{Evaluationsbeispiel 6 (Überkategorie: Mineralisch)}
	\label{t:evaluation-example6}
\end{table}

\begin{table}[h]
	
	\centering
	\begin{tabular}{|l|l|l|l|}
		\hline
		\textbf{Begriff} & \textbf{Erwartet} & \textbf{DBSCAN} & \textbf{OpenAI}\\ \hline
		 A W TR MW KS 17 5 & 4 & 4 & ~ \\ \hline
		FAS PUTZ TYP1 0 2 0 & 1 & 1 & ~ \\ \hline
		I BOD ESTRICH ZE 6 & 2 & 2 & 2 \\ \hline
		I BOD ESTRICH ZE 7 & 2 & 2 & 2 \\ \hline
		FAS PUTZ TYP1 0 2 & 1 & 1 & ~ \\ \hline
		I BOD BELAG BETONSTEIN 3 & 3 & 3 & 3 \\ \hline
		I BOD ESTRICH ZE 4 & 2 & 2 & 2 \\ \hline
		I BOD BELAG FLIESE 1 5 & 3 & 3 & 3 \\ \hline
		I BOD ESTRICH ZE 5 5 & 2 & 2 & 2 \\ \hline
		I W TR MW KS 11 5 & 4 & 4 & ~ \\ \hline
		A BOD KIES 5 & ~ & ~ & ~ \\ \hline
		Gefällebeton & 5 & ~ & ~ \\ \hline
		Beton & 5 & ~ & ~ \\ \hline
		Concrete & 5 & ~ & ~ \\ \hline
		\textbf{Auswertung} & richtig & teilweise & teilweise \\ \hline
	\end{tabular}
	\caption{Evaluationsbeispiel 7 (Überkategorie: Mineralisch)}
	\label{t:evaluation-example7}
\end{table}

\begin{table}[h]
	
	\centering
	\begin{tabular}{|l|l|l|l|}
		\hline
		\textbf{Begriff} & \textbf{Erwartet} & \textbf{DBSCAN} & \textbf{OpenAI}\\ \hline
		        Metall Edelstahl gebürstet & 1 & 1 & 1 \\ \hline
		Fensterbank Aussen Blech & ~ & ~ & -1 \\ \hline
		Metall Edelstahl Satiniert & 1 & 1 & 1 \\ \hline
		Metall Stahl matt & ~ & 1 & ~ \\ \hline
		\textbf{Auswertung} & richtig & falsch & falsch \\ \hline
	\end{tabular}
	\caption{Evaluationsbeispiel 8 (Überkategorie: Metall)}
	\label{t:evaluation-example8}
\end{table}

\begin{table}[h]
	
	\centering
	\begin{tabular}{|l|l|l|l|}
		\hline
		\textbf{Begriff} & \textbf{Erwartet} & \textbf{DBSCAN} & \textbf{OpenAI}\\ \hline
		   Metall Titanzink & ~ & 2 & ~ \\ \hline
		Metall Oberfläche lackiert elfenbein Mit Glaz & 1 & 1 & 1 \\ \hline
		Metall Oberfläche lackiert matt & 1 & 1 & 1 \\ \hline
		Metall Aluminium & ~ & 2 & ~ \\ \hline
		Metall lackiert & 1 & 1 & ~ \\ \hline
		Metall verzinkt & ~ & 2 & ~ \\ \hline
		\textbf{Auswertung} & richtig & richtig & teilweise \\ \hline
	\end{tabular}
	\caption{Evaluationsbeispiel 9 (Überkategorie: Metall)}
	\label{t:evaluation-example9}
\end{table}

\begin{table}[h]
	
	\centering
	\begin{tabular}{|l|l|l|l|}
		\hline
		\textbf{Begriff} & \textbf{Erwartet} & \textbf{DBSCAN} & \textbf{OpenAI}\\ \hline
		  Metall Edelstahl Satiniert & 1 & 1 & 1 \\ \hline
		Metall Edelstahl gebürstet & 1 & 1 & 1 \\ \hline
		Metall Stahl matt & 2 & 1 & ~ \\ \hline
		Metal Strugal Stainless Steel & 1 & 2 & ~ \\ \hline
		Aluminum Strugal RAL 9016 & ~ & 2 & ~ \\ \hline
		Fensterbank Aussen Blech & ~ & 2 & ~ \\ \hline
		\textbf{Auswertung} & richtig & falsch & teilweise \\ \hline
	\end{tabular}
	\caption{Evaluationsbeispiel 10 (Überkategorie: Metall)}
	\label{t:evaluation-example10}
\end{table}

\begin{table}[h]
	
	\centering
	\begin{tabular}{|l|l|l|l|}
		\hline
		\textbf{Begriff} & \textbf{Erwartet} & \textbf{DBSCAN} & \textbf{OpenAI}\\ \hline
		 StB C45 & 1 & 1 & 1 \\ \hline
		StB C55 & 1 & 1 & 1 \\ \hline
		StB C35 & 1 & 1 & 1 \\ \hline
		StB C30 & 1 & 1 & 1 \\ \hline
		StB C60 & 1 & 1 & 1 \\ \hline
		Fertigteil StB C35 & 1 & 1 & 1 \\ \hline
		StB FU & 1 & ~ & 1 \\ \hline
		\textbf{Auswertung} & richtig & teilweise & richtig \\ \hline
	\end{tabular}
	\caption{Evaluationsbeispiel 11 (Überkategorie: Verbundbauteil)}
	\label{t:evaluation-example11}
\end{table}

\begin{table}[h]
	
	\centering
	\begin{tabular}{|l|l|l|l|}
		\hline
		\textbf{Begriff} & \textbf{Erwartet} & \textbf{DBSCAN} & \textbf{OpenAI}\\ \hline
		 Aluminium & ~ & ~ & ~ \\ \hline
		Stahl verzinkt & ~ & ~ & ~ \\ \hline
		Edelstahl  & ~ & ~ & ~ \\ \hline
		\textbf{Auswertung} & richtig & richtig & richtig\\ \hline
	\end{tabular}
	\caption{Evaluationsbeispiel 12 (Überkategorie: Metall)}
	\label{t:evaluation-example12}
\end{table}

\begin{table}[h]
	
	\centering
	\begin{tabular}{|l|l|l|l|}
		\hline
		\textbf{Begriff} & \textbf{Erwartet} & \textbf{DBSCAN} & \textbf{OpenAI}\\ \hline
		  DA Flachdach Gefälledämmung & 2 & 2 & 1 \\ \hline
		Mineralwolle 4 & 1 & 1 & 1 \\ \hline
		WA XPS Dämmung 10cm & 3 & ~ & 1 \\ \hline
		XPS & 3 & 2 & 1 \\ \hline
		DA Flachdach Dämmung 14cm & 2 & 2 & 1 \\ \hline
		Mineralwolle 3 & 1 & 1 & 1 \\ \hline
		Brandriegel & ~ & ~ & ~ \\ \hline
		Trittschalldämmung  & ~ & ~ & ~ \\ \hline
		\textbf{Auswertung} & richtig & falsch & falsch \\ \hline
	\end{tabular}
	\caption{Evaluationsbeispiel 13 (Überkategorie: Dämmungen)}
	\label{t:evaluation-example13}
\end{table}

\begin{table}[h]
	
	\centering
	\begin{tabular}{|l|l|l|l|}
		\hline
		\textbf{Begriff} & \textbf{Erwartet} & \textbf{DBSCAN} & \textbf{OpenAI}\\ \hline
		   DA Flachdach Dämmung 14cm & 3 & 2 & 1 \\ \hline
		Mineralwolle 3 & 1 & 1 & 1 \\ \hline
		WA XPS Dämmung 10cm & 2 & 2 & 1 \\ \hline
		XPS & 2 & ~ & 1 \\ \hline
		DA Flachdach Gefälledämmung & 3 & 2 & 1 \\ \hline
		Mineralwolle 4 & 1 & 1 & 1 \\ \hline
		Trittschalldämmung & ~ & ~ & ~ \\ \hline
		Brandriegel  & ~ & ~ & ~ \\ \hline
		\textbf{Auswertung} & richtig & falsch & falsch \\ \hline
	\end{tabular}
	\caption{Evaluationsbeispiel 14 (Überkategorie: Dämmungen)}
	\label{t:evaluation-example14}
\end{table}

\begin{table}[h]
	
	\centering
	\begin{tabular}{|l|l|l|l|}
		\hline
		\textbf{Begriff} & \textbf{Erwartet} & \textbf{DBSCAN} & \textbf{OpenAI}\\ \hline
		  FAS WDVS WD EPS 8 & 1 & 1 & 1 \\ \hline
		FAS WDVS WD EPS 18 & 1 & 1 & 1 \\ \hline
		I BOD DAEM TD 3 & ~ & ~ & 2 \\ \hline
		I BOD DAEM WD 14 & 2 & 2 & 2 \\ \hline
		I BOD DAEM WD 15 & 2 & 2 & 2 \\ \hline
		A BOD DAEM XPS 11 & 4 & 4 & 4 \\ \hline
		A W NT DAEM PD XPS 12 & 3 & 3 & 3 \\ \hline
		A W NT DAEM PD XPS 6 & 3 & 3 & 3 \\ \hline
		A BOD DAEM XPS 12 & 4 & 4 & 4 \\ \hline
		A BOD DAEM XPS 10 & 4 & 4 & 4 \\ \hline
		\textbf{Auswertung} & richtig & richtig & teilweise \\ \hline
	\end{tabular}
	\caption{Evaluationsbeispiel 15 (Überkategorie: Dämmungen)}
	\label{t:evaluation-example15}
\end{table}

\begin{table}[h]
	
	\centering
	\begin{tabular}{|l|l|l|l|}
		\hline
		\textbf{Begriff} & \textbf{Erwartet} & \textbf{DBSCAN} & \textbf{OpenAI}\\ \hline
		Glas Isolierverglasung klar & 1 & 1 & 1 \\ \hline
		Glas Isolierverglasung 3 fach & 1 & 2 & 1 \\ \hline
		Glas Isolierverglasung 2 fach & 1 & 2 & 1 \\ \hline
		Glas klar & 1 & 1 & ~ \\ \hline
		Glas matt & ~ & ~ & ~ \\ \hline
		\textbf{Auswertung} & richtig & falsch & teilweise\\ \hline
	\end{tabular}
	\caption{Evaluationsbeispiel 16 (Überkategorie: Glas)}
	\label{t:evaluation-example16}
\end{table}

\begin{table}[h]
	
	\centering
	\begin{tabular}{|l|l|l|l|}
		\hline
		\textbf{Begriff} & \textbf{Erwartet} & \textbf{DBSCAN} & \textbf{OpenAI}\\ \hline
		Glas matt & ~ & ~ & ~ \\ \hline
		Glas Isolierverglasung Bronze & 1 & ~ & 1 \\ \hline
		Glass Strugal Clear Glazing & 1 & 2 & ~ \\ \hline
		Glas klar & 1 & 1 & ~ \\ \hline
		Glass Simonton Clear & 1 & 2 & ~ \\ \hline
		Glas Isolierverglasung klar & 1 & 1 & 1 \\ \hline
		\textbf{Auswertung} & richtig & falsch & teilweise \\ \hline
	\end{tabular}
	\caption{Evaluationsbeispiel 17 (Überkategorie: Glas)}
	\label{t:evaluation-example17}
\end{table}

\begin{table}[h]
	
	\centering
	\begin{tabular}{|l|l|l|l|}
		\hline
		\textbf{Begriff} & \textbf{Erwartet} & \textbf{DBSCAN} & \textbf{OpenAI}\\ \hline
		 Geschossdecke FB 200 Parkett & 2 & 2 & -1 \\ \hline
		 Holz generisch & ~ & 1 & ~ \\ \hline
		 Fußboden Parkett eiche & 2 & 2 & -1 \\ \hline
		 Dachdeckung Holz & 1 & 1 & 1 \\ \hline
		 Dachdeckung Holz 2 & 1 & 1 & 1 \\ \hline
		\textbf{Auswertung} & richtig & teilweise & falsch \\ \hline
	\end{tabular}
	\caption{Evaluationsbeispiel 18 (Überkategorie: Holz)}
	\label{t:evaluation-example18}
\end{table}

\begin{table}[h]
	
	\centering
	\begin{tabular}{|l|l|l|l|}
		\hline
		\textbf{Begriff} & \textbf{Erwartet} & \textbf{DBSCAN} & \textbf{OpenAI}\\ \hline
		 Holz Parkett & ~ & ~ & 1 \\ \hline
		Holz Birke & 1 & 1 & 1 \\ \hline
		Holz Eiche & 1 & 1 & 1 \\ \hline
		Standard Treppe & ~ & ~ & ~ \\ \hline
		\textbf{Auswertung} & richtig & richtig & teilweise \\ \hline
	\end{tabular}
	\caption{Evaluationsbeispiel 19 (Überkategorie: Holz)}
	\label{t:evaluation-example19}
\end{table}

\begin{table}[h]
	
	\centering
	\begin{tabular}{|l|l|l|l|}
		\hline
		\textbf{Begriff} & \textbf{Erwartet} & \textbf{DBSCAN} & \textbf{OpenAI}\\ \hline
		  Dachdeckung Holz & ~ & ~ & ~ \\ \hline
		Fußboden Terrasse Teakholz & ~ & ~ & ~ \\ \hline
		Holz HSB Steher & ~ & ~ & ~ \\ \hline
		Holz Astfichte senkrecht & ~ & ~ & ~ \\ \hline
		\textbf{Auswertung} & richtig & richtig & richtig \\ \hline
	\end{tabular}
	\caption{Evaluationsbeispiel 20 (Überkategorie: Holz)}
	\label{t:evaluation-example20}
\end{table}

\begin{table}[h]
	
	\centering
	\begin{tabular}{|l|l|l|l|}
		\hline
		\textbf{Begriff} & \textbf{Erwartet} & \textbf{DBSCAN} & \textbf{OpenAI}\\ \hline
		 Fußboden Epoxidharzbeschichtung & ~ & ~ & ~ \\ \hline
		BSt 500 M A & 1 & 1 & 1 \\ \hline
		BSt 500 M B & 1 & 1 & 1 \\ \hline
		\textbf{Auswertung} & richtig & richtig & richtig\\ \hline
	\end{tabular}
	\caption{Evaluationsbeispiel 21 (Überkategorie: Verbundbauteile)}
	\label{t:evaluation-example21}
\end{table}

\begin{table}[h]
	
	\centering
	\begin{tabular}{|l|l|l|l|}
		\hline
		\textbf{Begriff} & \textbf{Erwartet} & \textbf{DBSCAN} & \textbf{OpenAI}\\ \hline
		  A W TR STB OB 17 5 & 1 & 1 & -1 \\ \hline
		I W TR STB OB 25 & 1 & 2 & -1 \\ \hline
		A DE TR STB OB 30 & 3 & ~ & -1 \\ \hline
		I W TR STB OB 12 & 1 & 2 & -1 \\ \hline
		A W TR STB OB 25 & 1 & 1 & -1 \\ \hline
		A W TR STB OB 30 & 1 & 1 & -1 \\ \hline
		I W TR STB OB 22 & 1 & 2 & -1 \\ \hline
		I DE TR STB OB 20 & 3 & 3 & -1 \\ \hline
		I DE TR STB OB 45 & 3 & 3 & -1 \\ \hline
		I DE TR STB OB 25 & 3 & 3 & -1 \\ \hline
		\textbf{Auswertung} & richtig & teilweise & teilweise \\ \hline
	\end{tabular}
	\caption{Evaluationsbeispiel 22 (Überkategorie: Verbundbauteile)}
	\label{t:evaluation-example22}
\end{table}




