\chapter*{Abstract}
Anhand der vorliegenden wissenschaftlichen Arbeit soll ein Konzept für die hierarchische Strukturierung von Materialien aus \acf{ifc}-Dateien für eine Programmerweiterung der Bausoftware ORCA AVA erarbeitet werden. Dafür sollen passende \acf{nlp}-Algorithmen für die Strukturierung ausgewählt werden. Die Materialstrukturierung liefert folgende Probleme: Materialangaben bestehen meistens aus Stichpunkten mit nur einem oder wenigen Wörtern und fallen somit in die Kurztext-Klassifizierung. Zusätzlich müssen viele Fachbegriffe richtig verarbeitet werden. Es wurden verschiedene Algorithmen nach definierten Kriterien der Messbarkeit evaluiert und passende ausgewählt.

Die Materialstrukturierung wurde in die Textklassifizierung und eine weitere Feinstrukturierung aufgeteilt. Für die Textklassifizierung stellte sich ein Maximum Entropy Model mit dem \ac{sdca}-Optimierungsalgorithmus trotz Kurztext-Klassifizierung mit einer Genauigkeit von 88,2 \% als beste Wahl heraus. Bei der Feinstrukturierung wird eine domänenspezifisches \textit{fastText}-Modell trainiert, um die Bedeutung der Wörter, trotz Stichpunkten und Fachbegriffen, bei der Vektorisierung der Materialien mitzugeben. So können die Materialien mit \acf{dbscan}-Clustering weiter strukturiert werden. Dieses Vorgehen schneidet besser ab, als das Nutzen des fertig trainierten ChatGPT-Modells von OpenAI. Für das Ausführen der Strukturierung wurde ein Webservice implementiert, um die Machine-Learning-Modelle zentral nutzen zu können und keine Pythonumgebung für die ORCA AVA installiert werden muss.\\

\noindent Schlagworte: \acs{nlp}, Machine Learning, \acs{ifc}, \acs{bim}, ML.NET, fastText, \acs{dbscan}, OpenAI

