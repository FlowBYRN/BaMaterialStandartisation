\begin{onehalfspace}
\section{Motivation}
\subsection{Problemstellung}
\label{s:intro}
Das übergeordnete Ziel für die Bausoftware ORCA AVA ist es, möglichst viele Daten aus 3D-Modellen in der Ausschreibungssoftware automatisch importieren zu können. Ein Teil der aus den Modellen übernommen werden soll, sind Baustoffe und Materialien des Objektes. Das spart dem Architekten viel Zeit am Anfang in der Ausschreibung seines Projektes.
Als öffentliches Standardformat für 3D-Gebäudemodelle gibt es IFC, welches auch in der ORCA AVA benutzt wird. In diesen Modellen können auch die Materialien der einzelnen Bauteile spezifiziert werden. Diese können an verschiedenen Stellen an einem Bauteil im Modell angegeben werden. Für Materialbezeichungen gibt es auch keine richtigen Standards. Hier besteht das Problem, dass das Textfeldern für die Materialangabe ein offenes Textfeld ist und somit kein Standard existiert und jedes IFC-Modell anders strukturiert ist.


\subsection{Ziel}
Ziel ist es eine Kostengliederungsstruktur in der Bausoftware ORCA AVA aus den Materialeininformationen einer IFC-Datei zu generieren. Diese Kostengliederung kann am Anfang eines Projektes einmal importiert werden. Wenn man dann im laufe des Ausschreibungsprozesses ein Bauteil aus der IFC-Datei in die ORCA AVA übernimmt, wird es automatisch einem Material in dieser Kostengliederungsstruktur zugewiesen. So kann man die Ausschreibungspositionen nach Materialangaben auswerten.
\\


Ein Algorithmus soll zuerst die Möglichkeiten der Materialangabe zusammenführen. Außerdem handelt es sich bei der Materialangabe um ein Freitextfeld. Hier soll mit Hilfe von Natural Language Processing und Artificial Inteligence eine Lösung entwickelt werden, um eine klassifizierte und standardisierte Liste der Materialen zu erschaffen. 
\section{Forschungsstand}
\section{Vorläufige Gliederung}
\section{Zeitplan}

\begin{description}
	\item[1.]
	Recherchieren über verschiedene mögliche Techniken und Algorithmen
	\item[2.] 
	Austesten der gefundenen Algorithmen
	\item[3.]
	Messen der Ergebnisse der Algorithmen
	\item[4.] Implementieren des besten Ergebnisses
\end{description}

\section{Erste Literatur}
\end{onehalfspace}